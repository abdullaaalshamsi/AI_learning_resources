\documentclass[12pt]{book}

\usepackage[a4paper,margin=2.5cm]{geometry}

\usepackage{amsmath}
\usepackage{amssymb}
\usepackage{verbatim}

\begin{document}

\title{Tensor Mathematics Lessons}

\author{Abdulla Ali Alshamsi \\ ORCID:0009-0006-4905-5469}
\date{}
\maketitle

\tableofcontents

% =======================
% Introduction 
% =======================
\chapter{Introduction}
% [Write here: Introduction]

% =======================
% Chapter Example
% =======================
\chapter{T00}
% [Write here: Chapter overview]

% =======================
% Lesson Example
% =======================
\section{Lesson T00}
% [Write here: Lesson explanation]

% =======================
% Exercise E00
% =======================
\subsection*{Exercise E00 : Tensor Addition Law}

\paragraph{Understanding the Problem}
Tensor addition is a fundamental operation in tensor algebra.
Given two tensors $\mathbf{X}$ and $\mathbf{Y}$ of the same shape, the goal is
to compute a new tensor $\mathbf{Z}$ by adding corresponding elements from
$\mathbf{X}$ and $\mathbf{Y}$.
This operation is performed independently on each component of the tensors.

\paragraph{Mathematical Formulation}
Let
\[
\mathbf{X} = (x_1, x_2) \in \mathbb{R}^2,
\quad
\mathbf{Y} = (y_1, y_2) \in \mathbb{R}^2
\]

The tensor addition operator
\[
+ : \mathbb{R}^2 \times \mathbb{R}^2 \rightarrow \mathbb{R}^2
\]
is defined component-wise as:
\[
\mathbf{Z} = \mathbf{X} + \mathbf{Y}
=
(x_1 + y_1,\; x_2 + y_2)
\]

\paragraph{Mathematical Solution}
For numerical computation, each component of the output tensor is evaluated
independently according to the rule:
\[
z_i = x_i + y_i,\quad i = 1, 2
\]

This rule generalizes naturally to tensors of arbitrary dimension, since tensor
addition is always performed element-wise on corresponding positions.

\paragraph{Code Implementation}
\begin{verbatim}
def E00(x, y):
    # apply tensor addition law
    return z
\end{verbatim}



% =======================
% Exercise E01
% =======================
\subsection*{Exercise E01 : Tensor Subtraction Law}

\paragraph{Understanding the Problem}
Tensor subtraction is a fundamental operation derived from tensor addition.
Given two tensors $\mathbf{X}$ and $\mathbf{Y}$ of the same shape, the objective
is to compute a new tensor $\mathbf{Z}$ by subtracting each element of
$\mathbf{Y}$ from the corresponding element of $\mathbf{X}$.
This operation is performed independently on each component of the tensors.

\paragraph{Mathematical Formulation}
Let
\[
\mathbf{X} = (x_1, x_2) \in \mathbb{R}^2,
\quad
\mathbf{Y} = (y_1, y_2) \in \mathbb{R}^2
\]

The tensor subtraction operator
\[
- : \mathbb{R}^2 \times \mathbb{R}^2 \rightarrow \mathbb{R}^2
\]
is defined component-wise as:
\[
\mathbf{Z} = \mathbf{X} - \mathbf{Y}
=
(x_1 - y_1,\; x_2 - y_2)
\]

Tensor subtraction can also be interpreted using additive inverses:
\[
\mathbf{X} - \mathbf{Y} = \mathbf{X} + (-\mathbf{Y})
\]

\paragraph{Mathematical Solution}
For numerical computation, each component of the resulting tensor is evaluated
independently according to the rule:
\[
z_i = x_i - y_i,\quad i = 1, 2
\]

This rule generalizes naturally to tensors of arbitrary dimension, since tensor
subtraction is performed element-wise on corresponding positions.

\paragraph{Code Implementation}
\begin{verbatim}
def E01(x, y):
    # apply tensor subtraction law
    return z
\end{verbatim}

% =======================
% Exercise E01
% =======================

\subsection*{Exercise E02 : Tensor Hadamard Multiplication Law}

\paragraph{Understanding the Problem}
Tensor Hadamard multiplication is an element-wise multiplication operation.
Given two tensors $\mathbf{X}$ and $\mathbf{Y}$ of the same shape, the goal is
to compute a new tensor $\mathbf{Z}$ by multiplying each element of $\mathbf{X}$
with the corresponding element of $\mathbf{Y}$.
This operation is performed independently on each component of the tensors.

\paragraph{Mathematical Formulation}
Let
\[
\mathbf{X} = (x_1, x_2) \in \mathbb{R}^2,
\quad
\mathbf{Y} = (y_1, y_2) \in \mathbb{R}^2
\]

The Hadamard multiplication operator
\[
\odot : \mathbb{R}^2 \times \mathbb{R}^2 \rightarrow \mathbb{R}^2
\]
is defined component-wise as:
\[
\mathbf{Z} = \mathbf{X} \odot \mathbf{Y}
=
(x_1 y_1,\; x_2 y_2)
\]

\paragraph{Mathematical Solution}
For numerical computation, each component of the resulting tensor is evaluated
independently according to the rule:
\[
z_i = x_i \cdot y_i,\quad i = 1, 2
\]

This rule extends naturally to tensors of arbitrary dimension, since Hadamard
multiplication is always applied element-wise on corresponding positions.

\paragraph{Code Implementation}
\begin{verbatim}
def E02(x, y):
    # apply Hadamard multiplication law
    return z
\end{verbatim}






% =======================
% Conclusion
% =======================
\chapter{Conclusion}
% [Write here: Conclusion]

\end{document}
